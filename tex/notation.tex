\chapter{Notación}

\section{... de Lógica}

\begin{enumerate}
    \ii $\iff$ si y solo si
    \ii $\implies$ implica
\end{enumerate}

\section{... de Conjuntos}

\begin{enumerate}
    \ii $A \subset B$ $A$ es subconjunto de $B$
    \ii $A \subseteq B$ $A$ es subconjunto de $B$ o $B$
    \ii $A \setminus B$ $A$ sin $B$
    \ii $A \cap B$ la intersección de $A$ y $B$
    \ii $A \cup B$ la unión de $A$ y $B$
    \ii $a \in A$ el elemento $a$ pertenece al conjunto $A$
\end{enumerate}

\section{... de Geo}

\begin{enumerate}
    \ii $AB$ la recta que pasa por los puntos $A$ y $B$
    \ii $\seg{AB}$ el segmento de extremos $A$ y $B$
    \ii $\seg{AB}$ la longitud del segmento de extremos $A$ y $B$
    \ii $\angle ABC$ el ángulo formado por las rectas BA y CA
    \ii $\angle A$ el ángulo del vértice $A$
    \ii $\triangle ABC$ el triángulo que pasa por los puntos $A$, $B$ y $C$
    \ii $P_1P_2\dots P_n$ el polígono que pasa por los puntos $P_1$, $P_2$, $\dots$,  $P_n$
    \ii $\left[\triangle ABC\right]$ el área del triángulo que pasa por los puntos $A$, $B$ y $C$
    \ii $\left[P_1P_2\dots P_n\right]$ el área el polígono que pasa por los puntos $P_1$, $P_2$, $\dots$,  $P_n$
    \ii $\left(\triangle ABC\right)$ la circunferencia que pasa por los puntos $A$, $B$ y $C$
    \ii $\left(P_1P_2\dots P_n\right)$ la circunferencia que pasa por los puntos $P_1$, $P_2$, $\dots$,  $P_n$ (si es que existe)
    \ii $AB \parallel CD$ la recta $AB$ es paralela a la recta $CD$.
    \ii $AB \perp CD$ la recta $AB$ es perpendicular a la recta $CD$.
\end{enumerate}

\section{... Mixta}

\begin{enumerate}
    \ii $\forall$ para todo
    \ii $:$ tal que
    \ii $\exists$ existe
    \ii $\therefore$ concluimos
\end{enumerate}