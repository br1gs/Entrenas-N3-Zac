\chapter{Combo Dump}

\section{Conteo}

\begin{problem}[USAMO 2019/4, $3\clubsuit$]
  Let $n$ be a nonnegative integer.
  Determine the number of ways to choose sets
  $S_{ij} \subseteq \{1, 2, \dots, 2n\}$,
  for all $0 \le i \le n$ and $0 \le j \le n$
  (not necessarily distinct), such that
  \begin{itemize}
  \ii $|S_{ij}| = i+j$, and
  \ii $S_{ij} \subseteq S_{kl}$ if $0 \le i \le k \le n$
  and $0 \le j \le l \le n$.
  \end{itemize}
\end{problem}

Let's see each all the sets in a grid, $S_{i,j}$ will be in 
the $i$-th column (left to right) and $j$-th row (up to down). 
Let's fill first the $0,0$ to $n,n$ diagonal. We can do it 
in $\frac{2n!}{2^n}$ ways. Note all the other cells have 
$2$ options (if we fill it from left to right and up to down
in the upper part and from right to left and up to down in the 
bottom part). Giving us $\frac{2n!}{2^n} \times 
2^{((n+1)^2-(n+1))} = 2n!\cdot 2^{n^2}$ as the answer.

\section{Inducción} 
\begin{problem}[JMO 2020/1, $2\clubsuit$]
  Let $n \ge 2$ be an integer.
  Carl has $n$ books arranged on a bookshelf.
  Each book has a height and a width.
  No two books have the same height,
  and no two books have the same width.

  Initially, the books are arranged in
  increasing order of height from left to right.
  In a \emph{move}, Carl picks any two adjacent books
  where the left book is wider and shorter than the right book,
  and swaps their locations.
  Carl does this repeatedly until no further moves are possible.

  Prove that regardless of how Carl makes his moves,
  he must stop after a finite number of moves, and when he does stop,
  the books are sorted in increasing order of width from left to right.
\end{problem}

\begin{proof}
  We can easily use induction.
\end{proof}

\section{Invarianza}

\begin{sproblem}[Canada 2018/1, $3\clubsuit$]
  Consider an arrangement of tokens in the plane,
  not necessarily at distinct points.
  We are allowed to apply a sequence of moves of the following kind:
  select a pair of tokens at points $A$ and $B$ and move
  both of them to the midpoint of $A$ and $B$.
  We say that an arrangement of $n$ tokens is \emph{collapsible} if it is
  possible to end up with all $n$ tokens at the same point after a finite
  number of moves.

  Find all integers $n \ge 1$ such that:
  every arrangement of $n$ tokens is collapsible.
\end{sproblem}

\begin{claim}
  It is only possible for powers of $2$.
\end{claim}

\begin{proof}
    Note that for powers of $2$ we can apply induction easily since 
    we can just pair in any way the tokens at the beggining. \\
    To finish, we must prove that setting one token at $(1, 0)$ 
    and the rest at the origin is a counterexample. 
    It's easy to check that the $x$-coordinates 
    of all the tokens is always 
    a fraction with a power of $2$ as denominator, 
    and the sum of all of them is $1$. 
    The latter means that if it the set is collapsible, 
    the tokens must be at the $x$-coordinate $\frac 1n$ 
    which is impossible if $n$ isn't a power of $2$.
\end{proof}

\begin{problem}[ELMO 1999/2, $2\clubsuit$]
  Mr.\ Fat moves around on the lattice points
  according to the following rules:
  From point $(x,y)$ he may move to any of the points
  $(y,x)$, $(3x,-2y)$, $(-2x,3y)$, $(x+1,y+4)$ and $(x-1,y-4)$.
  Show that if he starts at $(0,1)$ he can never get to $(0,0)$.
\end{problem}

\begin{proof}
  With such possible movements, the sum of the coordinates is 
  never $0$ mod $5$.
\end{proof}

\begin{problem}[PAGMO 2021/1, $5\clubsuit$]
  There are $n \geq 2$ coins numbered from $1$ to $n$.
  These coins are placed around a circle, not necessarily in order.

  In each turn, if we are on the coin numbered $i$,
  we will jump to the one $i$ places from it,
  always in a clockwise order, beginning with coin number $1$.

  Find all values of $n$ for which there exists an arrangement
  of the coins in which every coin will be visited.
\end{problem} 

  \begin{claim}
    Only when $n$ is even.
  \end{claim}

  \begin{proof} 
    Firstly, note that the second time we land on a coin, 
    we have fallen on a loop. 
    Furthermore, we have to land at the coin numbered $n$ 
    when we already had visited the rest of coins since after the next turn, 
    we will land another time at the coin numbered $n$. 

    As a result of this, the first $n-1$ movements 
    will have a total lenght of $1+\dots+(n-1)=\frac{n(n-1)}{2}$. 
    If $n$ is odd, that sum will be a multiple of $n$ which means 
    we will land at the coin numbered $1$ generating a loop.

    If $n$ is even, it's easy to check that the following arrangement works:

    \begin{itemize}
      \ii Place the coin numbered $1$;

      \ii next to it, all the odd coins in ascending counterclockwise order;
      
      \ii on the other side of the coin numbered $1$, all the even coins 
      in descending clockwise order.
    \end{itemize}

  \end{proof}

\begin{problem}[Russia 2022/9.3, $3\clubsuit$]
  Suppose $200$ positive integers are written in a row.
  For any two adjacent numbers in a row, the right one is
  either $9$ times greater than or $2$ times smaller than the left one.
  Can the sum of these $200$ numbers equal $24^{2022}$?
\end{problem}

  Let the numbers in the row be $a_1$, \dots, $a_{200}$. 
  Note that $a_i \equiv 2 a_{i+1}$ mod $17$ always holds; thus, 
  $\sum a \equiv a_{200} (2^{200}-1) \equiv 24^{2022}$ mod $17$. 
  This is impossible because $17 \mid 2^{200}-1$ but not $24^{2022}$; 
  hence, a contradiction.

\begin{problem}[$3\clubsuit$]
  The numbers $1, 2, \dots, 1000$ are written on the board.
  In a turn, one can erase any two numbers $a$ and $b$
  and replace them with $ab$ and $a^2+b^2$.
  Prove that it is impossible to get at least $700$
  identical numbers by performing such operations.
\end{problem}

\begin{proof}
  Checking that the number of multiples of $3$ in the board 
  is invariant, is pretty straightforward; therefore, at most 
  667 numbers can be identical.
\end{proof}

\begin{problem}[USAMTS 4/1/31, $5\clubsuit$]
  A group of $100$ friends stands in a circle.
  Initially, one person has $2019$ mangoes, and no one else has mangoes.
  The friends split the mangoes according to the following rules:
  \begin{itemize}
  \ii sharing: to share, a friend passes two mangoes to the left
  and one mango to the right.
  \ii eating: the mangoes must also be eaten and enjoyed.
  However, no friend wants to be selfish and eat too many mangoes.
  Every time a person eats a mango, they must also pass another mango to the right.
  \end{itemize}
  A person may only share if they have at least three mangoes,
  and they may only eat if they have at least two mangoes.
  The friends continue sharing and eating,
  until so many mangoes have been eaten that no one is able to share or eat anymore.
  Show that there are exactly eight people stuck with mangoes,
  which can no longer be shared or eaten.
\end{problem} 

\begin{proof}
  Label the friends from the person with $2019$ mangoes 
  (labeled  with $0$) to the right. Let $m_i$ be the number of 
  mangoes the person labeled with $i$ has. Note that the residue 
  of $\sum 2^im_i$ mod $2^{100} - 1$ is invariant; hence, always 
  2019. The sharing ends when each friend has $1$ or $0$ mangoes. 
  Over this conditions, the only way of making the defined sum 
  $\equiv 2019$ mod $2^{100}-1$ is to represent $2019$ in it's 
  binary unique representation which has $8$ $1$'s.
\end{proof}

\section{Monovarianza}

\begin{problem}[Canada 2012, $5\clubsuit$]
  A number of robots are placed on the squares of a finite, rectangular
  grid of squares. A square can hold any number of robots. Every edge of
  each square of the grid is classified as either passable or impassable.
  All edges on the boundary of the grid are impassable. You can give any
  of the commands up, down, left, or right.

  All of the robots then simultaneously try to move in the specified
  direction. If the edge adjacent to a robot in that direction is
  passable, the robot moves across the edge and into the next square.
  Otherwise, the robot remains on its current square. You can then give
  another command of up, down, left, or right, then another, for as long
  as you want. Suppose that for any individual robot, and any square on
  the grid, there is a finite sequence of commands that will move that
  robot to that square. Prove that you can also give a finite sequence of
  commands such that all of the robots end up on the same square at the
  same time.
\end{problem}

\begin{claim}
  Two robots can meet in a finite number of commands.
\end{claim}

\begin{proof}
  Let their minimum distance be the shortest path for 
  getting from the initial square of $R_1$ to the initial 
  square of $R_2$. Make $R_1$ follow this path. Note that 
  the new minimum distance between the same robots is greater 
  or equal than the old one. Since the grid is finite, if we  
  keep repeting this process, the minimum distance 
  will keep decreasiing (otherwise the second robot 
  would be pushen out of the limits of the grid); hence, 
  the minimum distance will eventually became $0$.
\end{proof}

Repeat this process and meet robots $1$ by $1$ (if two robots 
meet, we can just focus in one robot since the other will 
do exactly the same).

\section{Recursión}

\begin{problem}[USAMO 1996/4, $3\clubsuit$]
  An $n$-term sequence $(x_1, x_2, \dots, x_n)$
  in which each term is either $0$ or $1$ is called a binary sequence of length $n$.
  Let $a_n$ be the number of binary sequences of length $n$ containing
  no three consecutive terms equal to $0$, $1$, $0$ in that order.
  Let $b_n$ be the number of binary sequences of length $n$ that
  contain no four consecutive terms equal to 0, 0, 1, 1 or 1, 1, 0, 0 in that order.
  Prove that $b_{n+1} = 2a_n$ for all positive integers $n$.
\end{problem}

\begin{proof}
  Note that if we translate a binary string and it's inverse 
  to gray code, the result will be the same except by the first 
  digit. And, a string of lenght $n+1$ that contains $0011$ or 
  $1100$ in gray code, will always have a $010$ substring in 
  it's last $n$ digits. Thus, $\frac{2^{n+1}-b_{n+1}}{2} = 2^{n}-a_n$ 
  which finishes the problem.
\end{proof} 

\section{Construcciones}

Vamos a jugar minecraft muchachos.

\begin{problem}[Tuymaada 2018/J6, $3\clubsuit$]
  The numbers $1, 2, \dots, 1024$ are written on a blackboard.
  The following procedure is performed ten times:
  partition the numbers on the board into disjoint pairs,
  and replace each pair with its nonnegative difference.
  Determine all possible values of the final number.
\end{problem}

\begin{claim}
  If there is an even number of odd numbers, 
  after applying the procedure, the number of odd numbers is 
  still even. 
\end{claim}

\begin{proof}
  Let's try to demonstrate the contrary. For that, 
  take a fixed odd number of even-odd pairs (since it's the only 
  way to get an odd difference) and pair the remaining 
  odd numbers with another odd number. Note that we have an 
  odd number of odd numbers remaining so the later is impossible.
\end{proof}

Since we start with an even number of odd numbers, the claim 
implies the final number is always even (since 1 isn't even). 

\begin{claim}
  The answer is all even numbers between $0$ and $1022$.
\end{claim}

\begin{proof}
  Obviously an even number $\ge 1024$ doesn't work. 
  Use the following construction for obtaining the desired 
  even difference, $2k$:

  \begin{enumerate}
    \ii Pair $(1, 2 + 2k)$ and consecutive numbers. 
    \ii After the first procedure, we will get the numbers 
    $1+2k$ $1$, $\dots$, $1$.
    \ii Pair in any way you want.
    \ii It will result in $2k$, $0$, $\dots$, $0$.
    \ii Repeat $3$ and $4$ until all the $0$'s dissapear.
  \end{enumerate}
\end{proof}

\begin{problem}[USAMO 2021/4, $5\clubsuit$]
  A finite set $S$ of positive integers has the property that,
  for each $s\in S$, and each positive integer divisor $d$ of $s$,
  there exists a unique element $t\in S$ satisfying $\gcd(s,t) = d$.
  (The elements $s$ and $t$ could be equal.)

  Given this information, find all possible values for the
  number of elements of $S$.
\end{problem}

It's easy to see that the number of divisors of each element 
of $S$ is the cardinality of the set.

\begin{claim}
  if $p \mid s \implies p^2 \nmid s$ for every prime $p$.  
\end{claim}

\begin{proof}
  Suppose the contrary, it exists a prime $p$ such that 
  $p^{\alpha} \mid s$ for an $\alpha \ge 2$. 
  Note that an element has to be of the form 
  $k\frac{s}{p}$ with a $k$ such that $(k, s) = 1$ in order to
  get $\frac{s}{p}$ when pairing it with $s$. Let the function $d$ 
  denote the number of divisors of a number $\implies 
  d(s) = d(k\frac{s}{p}) = 
  d(k) \cdot d(\frac{s}{p}) = d(k) \cdot d(s) \cdot 
  \frac{\alpha}{\alpha + 1} \implies 1 = d(k) \cdot 
  \frac{\alpha}{\alpha + 1} \implies  d(k) = 
  \frac{\alpha + 1}{\alpha}$. Hence, a contradiction because 
  $\frac{\alpha + 1}{\alpha}$ isn't an integer.
\end{proof}

Furthermore, every $s$ is formed by the product of $x$ distinct 
primes for a fixed $x$.

\begin{claim}
  $0$ and the non-negative powers of $2$ are all the 
  possible cardinalities.
\end{claim}

\begin{proof}
  We will proceed by induction. Obiously $0$ works. 
  When the cardinality is $1$, $S = {1}$ works, that is our 
  inductive base. Assume that it works for the set $S$ whose 
  cardinality is $2^n$. Then, we can create a new set 
  $S' = S\times p \cup S\times q$ where $p$ and $q$ are primes 
  which aren't factors of any of the element of $S$. 
  For each $s \in S$, if we pair (by pairing I refer to 
  computing the gcd of two numbers) $sp$ with each element of 
  $S'$ we will get all it's divisors since by pairing it 
  to an element of $S \times q$ we get all the divisors of $s$ 
  and by pairing it with an element of $S\times p$ we get 
  all the divisors of $s$ multiplied by $p$ which are all the 
  factors of $sp$. This makes $S'$ fullfill the condition with 
  $2^{n+1}$ elements completing our inductive step. 
\end{proof}

\begin{problem}[USAMO 2022/5, $9\clubsuit$]
  A function $f \colon \RR \to \RR$ is
  \emph{essentially increasing}
  if $f(s) \leq f(t)$ holds
  whenever $s\leq t$ are real numbers such that $f(s)\neq 0$ and $f(t)\neq 0$.

  Find the smallest integer $k$ such that
  for any $2022$ real numbers $x_1$, $x_2$, \dots, $x_{2022}$,
  there exist $k$ essentially increasing functions $f_1, \dots, f_k$ such that
  \[ f_1(n) + f_2(n) + \dotsb + f_k(n) = x_n
  \qquad \hbox{ for every } n = 1, 2, \dots, 2022. \]
\end{problem}

Consider a non-fixed $m$ instead of $2022$.

\begin{claim}
  $k \ge \ceiling{log_2 m}$.
\end{claim}

\begin{proof}
  Suppose the contrary. Take extrictly decreasing values for 
  the $x$'s. There are at most $2^{k}$ distict binary strings 
  for the $m$ possible values of $n$ (A $0$ in the $x$-th 
  position will represent $f_x = 0$ and a $1$ every other number); 
  hence one binary string appears twice by the pigeonhole 
  principle. Say the $x$'s with the same string are $x_a$,  $x_b$. 
  WLOG $a > b \implies \sum f_i(a) > \sum f_i(b)$; hence 
  a contradiction since the functions are esentially increasing.
\end{proof}

We can construct inductively an arrangemnent for 
$k = \ceiling{log_2 m}$ as it follows 
(k = 3 in the following example): 
\begin{center}
  \begin{tabular}{c|c|c|c|}
    $n$ & $f_1(n)$ & $f_2(n)$ & $f_3(n)$ \\
    \hline 
    $1$ &  &  & 0 \\
    \hline
    $2$ &  &  & 0 \\  
    \hline
    $3$ &  &  & 0 \\
    \hline
    $4$ & $2C$ & $C$ & $x_4 -3C$ \\
    \hline
    $5$ & $2C$ & 0 & $x_5 -2C$ \\
    \hline
    $6$ & 0 & $C$ & $x_6 -C$ \\
    \hline
    $7$ & 0 & 0 & $x_7$ \\
    \hline
  \end{tabular}
\end{center}

Fill the blank spaces with the previous $k$ and choose a 
gigantic constant $C$ the $f$'s keep being essentially 
increasing.

\section{Juegos}

\begin{problem}[ELMO SL 2019 C1, $3\clubsuit$]
  Let $n \ge 3$ be a fixed positive integer.
  Elmo is playing a game with his clone.
  Initially, $n\geq 3$ points are given on a circle.
  On a player's turn, that player must draw a triangle
  using three unused points as vertices, without creating any crossing edges.
  The first player who cannot move loses.
  If Elmo's clone goes first and players alternate turns,
  which player wins for each $n$?
\end{problem}

If $n$ is odd, the clone can make a triangle such that in one arc has no 
points and in the other two have the exact same amount of points; 
if $n$ is even, make a triangle such that in one arc has 
one point and in the other two arcs have the exact same amount 
of points. Note that from then, the clone can reflect Elmo's 
triangles ensuring his win for all $n$.

\begin{problem}[Shortlist 2009 C1, $9\clubsuit$]
  Consider $2009$ cards, each having one gold side and one black side,
  lying on parallel on a long table.
  Initially all cards show their gold sides.
  Two players, standing by the same long side of the table,
  play a game with alternating moves.
  Each move consists of choosing a block of $50$ consecutive cards,
  the leftmost of which is showing gold, and
  turning them all over,
  so those which showed gold now show black and vice versa.
  The last player who can make a legal move wins.

  \begin{enumerate}[(a)]
  \ii Does the game necessarily end?
  \ii Does there exist a winning strategy for the starting player?
  \end{enumerate}
\end{problem}

  The game always end since we can translate the colour of the 
  cards into a binary string which always reduces. 

  From here, label the cards from left to right. 
  Note that in each turn, exactly one of the cards labeled with 
  $\{10, 60, \dots, 1960\}$ increases or decreases by 
  exactly one. Since there are $40$ elements in that set, and 
  all the cards in the set need to be black for the game to 
  finish, the game needs an even number of turns to reach an end; 
  hence, player one can't win. 

\section{ndump}

\begin{problem}[PAGMO 2021/4, $2\clubsuit$]
  Lucy multiplies some positive one-digit numbers (not necessarily distinct)
  and obtains a number $n$ greater than $10$.
  Then, she multiplies all the digits of $n$ and obtains an odd number.
  Find all possible values of the units digit of $n$.
\end{problem}

  $5$ works since $3 \times 5 = 15$. Of course any even digit 
  can't divide $n$. Then, assume $n = 3^a \times 7^b$. 
  Note that $3^a \times 7^b \equiv 21^{b} \times 3^{a-b} \equiv 
  3^{a-b}$ mod $20$. Every power of $3$ has a residue smaller 
  than $10$ mod $20$; thus, it's impossible for every digit 
  other than $5$.  
