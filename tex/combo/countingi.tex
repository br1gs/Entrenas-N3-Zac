\chapter{Conteo I, el Comienzo}

¡Bienvenido seas al área que en mi opinión representa mejor a la olimpiada! 
Aquella donde la creatividad siempre está presente y las soluciones 
hermosas abundan. Empecemos con un problema:

Una araña fue invitada a la boda de su amiga la hormiga. 
En la mañana antes del evento, se vió en un gran problema: 
vestirse. Como quiere ir formal (y además sentirse bonita), 
se pondrá mallas y tacones en cada una de las ocho patas. 
Si todas las mallas, al igual que los tacones, son 
indistinguibles entre si, y además debe ponerse la malla 
antes del tacón en cada pata, ¿de cuántos órdenes distintos 
se puede vestir?

Revisemos un poco de teoría para poder resolver esta belleza :)

\section{Dos Principios Fundamentales}

\begin{example}
    Un platillo en un restaurante consta de una sopa y una 
    ensalada. Si ofrecen $5$ sopas distintas y $4$ ensaladas 
    distintas, ¿de cuántas formas puedes pedir tu platillo?
\end{example}

El \textbf{principio fundamental del conteo (PFC)} dice que si 
hay $n$ maneras de hacer una tarea y $m$ maneras de hacer una 
segunda tarea, entonces el total de formas distintas de hacer 
ambas tareas a la vez es $n \times m$.

\begin{example}
    En el mismo restaurante, ofrecen de postre helado o una 
    rebanada de pastel (no se permite ordenar ambos). Si hay $7$ 
    sabores de helado y $3$ pasteles distintos, ¿de cuántas 
    formas puedes pedir tu postre? 
\end{example}

El \textbf{principio de la suma (PS)} dice que si 
hay $n$ maneras de hacer una tarea y  hay otras $m$ maneras de 
hacer esa misma tarea, entonces el total de formas distintas de 
hacer dicha tarea es $n + m$.

Muchos problemas de combinatoria se pueden resolver aplicando 
estos dos simples principios. Incluso fórmulas más avanzadas 
tan solo son extensiones de estas reglas.

\section{Permutaciones}

\begin{definition}[$n!$]
    Leido "$n$ factorial", es el producto de todos los enteros 
    positivos menores o iguales al entero positivo $n$.
\end{definition}

\begin{definition}[Permutación]
    Es un reacomodo donde importa el orden.
\end{definition}

\begin{example}
    ¿De cuántas formas se pueden permutar las cinco letras 
    $ABCDE$?
\end{example}

\begin{theorem}[Permutaciones de los $n$ objetos disponibles]
    Dados $n$ objetos distintos, la cantidad de permutaciones 
    entre sí es igual a 
    \[n!.\]
\end{theorem}

\begin{example}
    ¿Cuántas palabras distintas de tres letras se pueden 
    obtener al permutar las letras de la palabra $FERMAT$?
\end{example}

\begin{theorem}[Permutaciones de $k$ objetos de los $n$ disponibles]
    Dados $n$ objetos disponibles donde se quiere permutar $k$ 
    de ellos, la cantidad de maneras de hacer esto es 
    \[\frac{n!}{(n-k)!}.\]
\end{theorem}

Pero, ¿que hay de la palabra $GAUSS$? Al inicio podrías pensar 
que la respuesta es $5!$, pero es en realidad $\frac{5!}{2}$.

\begin{question}
    ¿Por qué?    
\end{question}

Más allá, pensarías que la palabra $LEGENDRE$ tiene 
$\frac{8!}{3}$ formas de permutarse, pero son $\frac{8!}{6}$.

\begin{question}
    ¿Por qué?    
\end{question}

\begin{example}
    ¿De cuántas formas se pueden permutar las letras de la 
    palabra $MISSISSIPPI$
\end{example}

El ejemplo de arriba es un problemas de permutaciones con 
repeticiones, la formula es fácil de recordar haciendo ejemplos, 
pero escrita formalmente es:

\begin{theorem}[Pemutaciones con repeticiones]
    Dada una colección $n$ pelotas que son indistinguibles 
    excepto por su color, si hay $a_i$ pelotas del color $i$, 
    entonces el número de formas de acomodar las pelotas en 
    una fila es de 
    \[
    \frac{\left(a_1+a_2+\dots+a_n\right)!}{a_1!a_2!\dots a_n!}.
    \]
\end{theorem}

\section{Bonus}

\subsection{Solución al Problema de la Araña}

Resolvamos el problema del inicio. Asígnemosle una letra a 
cada pata de la señorita araña; digamos que las letras son 
$ABCDEFGH$. Nótese que cada "orden" que siga la araña 
se puede traducir a una cadena de instrucciones donde la 
aparición de la instrucción $X$ significa que se va a poner una 
prenda en la pata a la que se le fue asignada aquella letra. Nótese 
que cada $X$ aparece exactamente dos veces, puesto 
que la primera vez que aparece, se pone la malla y la segunda, 
el tacón. Entonces, nuestra respuesta es la cantidad de 
permutaciones de la cadena $AABBCCDDEEFFGGHH$, que es 
\[\frac{16!}{2^8}.\]

La idea de transformar un problema a otro equivalente es 
ampliamente utilizada en la combinatoria de olimpiadas. 
Particularmente, veremos otros ejemplos muy interesantes de 
traducir problemas a cadenas de caracteres la siguiente clase.   

\subsection{Datos Extra}

Al principio fundamental del conteo también se le conoce como 
principio de la multiplicación o principio multiplicativo. 
Mientras que al principio de la suma también se le conoce como 
principio aditivo.

Por acuerdo, el valor de $0!$ es $1$.

Un buen razonamiento combinatórico depende MUCHOO de saber 
cuando sumar, multiplicar o dividir.

\begin{moral}
    ¡CUIDA NO CONTAR DE MÁS!
\end{moral}

\newpage

\section{Problemas}

Cada problema que resuelvas te dará el número de treboles 
que especifica ($x \clubsuit$), ¡colecta los más que puedas!

\epigraph{Todo está muchas veces, catorce veces, pero dos cosas hay en el 
mundo que parecen estar una sola vez: arriba, el intricado sol; abajo, 
Asterión}{Jorge Luis Borges, en La casa de Asterión}

\begin{problem}[$2 \clubsuit$]
    ¿Cuántos números positivos de 3 dígitos hay? ¿Cuántos 
    números enteros de tres o menos cifras hay?
\end{problem}

\begin{problem}[$2 \clubsuit$]
    Se quiere escoger un libro de entre 3 materias: matemáticas, historia 
    y biología. Hay 6 libros de Matemáticas, 9 de Historia y 4 de Biología. 
    ¿Cuántas opciones para escoger un libro tenemos?
\end{problem}

\begin{problem}[$2 \clubsuit$]
    ¿Cuántas contraseñas de $4$ dígitos se pueden formar?
\end{problem}

\begin{problem}[$2 \clubsuit$]
    ¿Cuántas palabras se pueden escribir usando todas las letras de la 
    palabra ZACATECAS?
\end{problem}

\begin{problem}[$2 \clubsuit$]
    ¿De cuántas formas puedes acomodar $5$ libros en un 
    estante?
\end{problem}

\begin{problem}[$2 \clubsuit$]
    ¿De cuántas formas se pueden acomodar 3 libros de álgebra, 
    2 de combinatoria y 3 de teoría de números en un estante?
\end{problem}

\begin{problem}[$3 \clubsuit$]
    En una examen hubo $6$ problemas. Si cada concursante 
    los resolvió en un orden distinto, ¿cuál es el máximo 
    numero de competidores en aquella olimpiada?
\end{problem}

\begin{problem}[$2 \clubsuit$]
    ¿Cuántas palabras de seis letras se pueden formar si 
    solo se dispone de un alfabeto con dos letras: a y b?
\end{problem}

\begin{dproblem}[$3 \clubsuit$]
    ¿Cuántos subconjuntos tiene un conjunto de $n$ elementos?
\end{dproblem}

\begin{problem}[OMMEB 2017, $3 \clubsuit$]
    Dada la lista de números \(1, 2, 3, 4, 5, 6, 7, 8, 9\) 
    una sublista se forma tomando al menos un número de la 
    lista y ordenar de menor a mayor. Por ejemplo \(1, 2, 8\) 
    es una sublista. Encuentra la cantidad de sublistas en las 
    que ninguno de los números \(2, 3, 5\) o \(7\) aparecen.
\end{problem}

\begin{problem}[$2 \clubsuit$]
    De cuántas formas se puede escoger un presidente, 
    vicepresidente y secretario en un grupo de $7$ personas.
\end{problem}

\begin{problem}[$3 \clubsuit$]
    ¿Cuántos números de $4$ dígitos hay tales que el primer 
    dígito es impar y el resto pares si todos los dígitos 
    son distintos entre si?
\end{problem}

\begin{problem}[$2 \clubsuit$]
    ¿Cuántos números capicua de $5$ dígitos hay? Un número 
    capicua es aquel que se lee igual de izquierda a 
    derecha que de derecha a izquierda.
\end{problem}

\begin{problem}[$4 \clubsuit$]
    Al cumpleañero de la fiesta anterior ya le compraste su 
    regalo, pero falta la envoltura. La caja en la que irá puede 
    ser chica, mediana o grande. Si se elige la chica, ésta 
    puede ser de 5 colores diferentes; si se elige la mediana, 
    de 3 y si se elige la grande, de 6. Si independientemente 
    del tamaño, la caja puede llevar o no un moño, ¿de cuántas 
    formas diferentes puede ser la envoltura del regalo?
\end{problem}

\begin{problem}[$4 \clubsuit$]
    Las ciudades de Nápoles, Venecia, Roma y Florencia están 
    unidas entre ellas. A cada dos de ellas las unen 7 caminos 
    diferentes. ¿De cuántas formas se puede ir de Venecia a
    Florencia sin pasar dos veces por la misma ciudad?
\end{problem}

\begin{problem}[$6 \clubsuit$]
    Claudia quiere pintar las 5 paredes de su habitación. 
    Ha comprado 12 colores diferentes de pintura. ¿De cuántas 
    formas puede pintar su habitación? ¿Y si no quiere que 2 
    paredes juntas tengan el mismo color? ¿Y si no quiere 2 
    paredes del mismo color?
\end{problem}

\begin{problem}[$6 \clubsuit$]
    La calculadora de Ale perdió las teclas 0, 1 y 2. En ella, 
    puede escribir un número de hasta 8 dígitos. ¿Cuántos 
    números distintos puede escribir en ella? ¿Y si quiere 
    escribir un número de 6 dígitos que tenga un único dígito 
    5? ¿Y si quiere uno de 4 dígitos que tenga un único 5 y un 
    único 4?
\end{problem}

\begin{problem}[$4 \clubsuit$]
    A, B, C y D toman cada uno una ficha distinta de dominó. 
    Notan que la cantidad de puntos en cada ficha que tomaron 
    es la misma. ¿De cuántas formas es esto posible?
\end{problem}

\begin{problem}[OMMEB 2019, $3 \clubsuit$]
    Un píxel está formado por $3$ leds: uno rojo, uno verde y uno azul, 
    donde cada uno puede prender con $4$ intensidades de luminosidad 
    diferentes (además de que pueden estar apagados). Cada configuración 
    de estos tres leds determina el color del píxel. ¿Cuántas 
    configuraciones hay con el led azul prendido?
\end{problem}

\begin{problem}[$4 \clubsuit$]
    ¿Cuántos números de 20 cifras se pueden formar utilizando los dígitos 
    $1$, $2$ y $3$ si la diferencia entre dos cifras consecutivas debe ser 
    siempre exactamente de $1$?
\end{problem}

\begin{problem}[$4 \clubsuit$]
    ¿Cuántos números de 4 cifras tienen exactamente un 8 entre sus dígitos?
    ¿Cuántos tienen al menos un 8?
\end{problem}

\begin{problem}[$3 \clubsuit$]
    ¿De cuántas maneras se pueden colocar ocho torres en un tablero de 
    ajedrez de manera que no se ataquen entre ellas?
\end{problem}

\begin{problem}[$4 \clubsuit$]
    Edgardo y sus otros $5$ amigos van al cine. Al llegar a la sala, 
    se percatan de que hay $12$ filas de $7$ asientos cada una. Si quieren 
    todos sentarse en una misma fila, pero Edgardo no puede sentarse al 
    lado de Rodrigo, ¿de cuántas formas pueden escoger sus asientos?
\end{problem}

\begin{problem}[OMMEB 2017, $4 \clubsuit$]
    Encuentra la cantidad de enteros positivos de cinco dígitos 
    distintos tales que cada uno de sus tres dígitos intermedios 
    es igual al promedio de sus dos dígitos adyacentes. 
    Un ejemplo de estos números es \(12345\).
\end{problem}

\begin{problem}[OMMEB 2017, $6 \clubsuit$]
    Sea $M$ el conjunto \(1, 2, 3, \ldots , 2017\). Para cada 
    subconjunto $A$ de $M $se denota por $S_A$ a la suma de 
    todos los elementos de $A$. Calcula el promedio de todos 
    los números $S_A$.
\end{problem}

\begin{problem}[USAMO 2019/4, $7\clubsuit$]
    \jp

    Sea $n$ un entero no negativo.
    Determina el número de formas de escoger los conjuntos
    $S_{ij} \subseteq \{1, 2, \dots, 2n\}$,
    para todo $0 \le i \le n$ y $0 \le j \le n$
    (no necesariamente distintos), tal que
    \begin{itemize}
    \ii $|S_{ij}| = i+j$, y
    \ii $S_{ij} \subseteq S_{kl}$ si $0 \le i \le k \le n$
    y $0 \le j \le l \le n$.
    \end{itemize}
\end{problem}

\noindent El máximo número de $\clubsuit$ en este capítulo es de 
$89 \clubsuit$.
