\chapter{Fundamentos Algebráicos}

Si me preguntasen: "¿Cuál es el área que se cuela más en las 
otras áreas de olimpiada. Sin dudarlo, el álgebra sería mi 
respuesta. Es un hecho, nos guste o no. 
Utilizar los principios del álgebra es muuuyy frecuente 
en cualquier problema de olimpiada (incluso puedes aplicarle 
a un problema de geo una metamorfosis que lo convierte en 
álgebra pura). Por esto, hay que atornillar en nuestro 
cerebro algunos principios y leyes básicas del álgebra 
(tampoco son tan difícile de recordar eh). ¡También debes dominar 
las ecuaciones como un experto!

En el álgebra se utilizan símbolos (llamados variables) que por 
lo general son letras del alfabeto para representar números cuyo 
valor no es "explícito". Puede que ser fijo o puede no serlo.

Nosotros también podemos usar las variable a nuestra 
conveniencia sustituyendo un valor que desconocemos por una 
letra.  

\begin{example}
Expresa un número más tres en lenguaje algebraico 
\end{example}

La expresión que buscamos es:
\[x+3\]

Es muy importante dominar esta 
intuición para poder traducir los problemas del 
español al álgebra y así resolverlos. 

Trabajar con lenguaje algebráico al inicio es muy confuso, 
pues la forma en la que se comunican los números es anormal 
para nosotros. 

Empezemos a revisar los preliminares. 
¿Listo para la abstracción?

\section{Definiciones Elementales}

El conjunto de números que se utiliza para contar se 
le conoce como el conjunto de \textbf{números naturales} 
y se denota por: 

\[\NN = \{1, 2, 3, \dots\}\]

Rara vez se suele considerar al $0$ dentro de los naturales 
pero yo jamás lo haré (y en un examen te lo deben especificar, 
igual si te queda la duda, ¡siempre puedes preguntar!).

La operación básica entre los números es la \textbf{suma} 
cuyo operador es $+$.

\begin{definition}[Neutro aditivo]
    Aquel número que al sumarse con cualquier otro no modifica 
    el valor de este último.
\end{definition}

\begin{question}
    ¿Qué número es el neutro aditivo?
\end{question}

\begin{definition}[Inverso aditivo]
    Para cualquier número, su inverso aditivo es el 
    número que se le debe sumar para obtener como resultado 
    cero.
\end{definition}

Nótese que el inverso aditivo de un número natural siempre es 
negativo, es decir, no está dentro del conjunto de los números 
naturales. Por esto debemos extender el conjunto de los 
números naturales agregándole sus inversos aditivos 
y el neutro aditivo, a este conjunto se le conoce como el conjunto 
de los números enteros y se denota por:

\[\ZZ = \{\dots, -3, -2, -1, 0, 1, 2, 3, \dots\}\]

\begin{moral}
    El $0$ no es ni positivo ni negativo, es neutro.
\end{moral}

\begin{definition}[Operaciones inversas]
    A dos operaciones se les conoce como inversas 
    si cuando se le aplican a un número 
    (en algún orden) el número inicial no cambia.
\end{definition}

Cabe recalcar que a la operación inversa a la suma se le conoce 
como \textbf{resta}. Cada vez que restamos un número, realmente le 
estamos sumando el inverso aditivo del números. El operador 
de la resta es $-$.

Con forme empezaron a usarse las matemáticas formalmente, 
se empezaron a presentar varias operaciones de la siguiente forma: 
"Un número $a$ es sumado $b$ veces." Por conveniencia, 
decidieron darle nombre a esta operación; le llamaron 
\textbf{multiplicación} y se le otorgó el operador 
$\times$, $\cdot$, $(\dots)(\dots)$.

\[ab=\underbrace{a+a+\dots+a+a}_{b}\]

\begin{definition}[Neutro multiplicativo]
    Aquel número que al multiplicarse con cualquier otro no 
    modifica el valor de este último.
\end{definition}

\begin{question}
    ¿Qué número es el neutro multiplicativo?
\end{question}

\begin{definition}[Inverso multiplicativo]
    Para cualquier número, su inverso multiplicativo es el 
    número por el que se debe multiplicar para obtener como 
    resultado uno.
\end{definition}

Nótese que el inverso multiplicativo de un número entero 
nunca es entero, es en realidad una fracción. Ante la 
existencia de las fracciones se creó un nuevo conjunto 
que abarca los enteros y las fracciones, a este conjunto 
se le conoce como el conjunto de los números racionales y 
se denota por 

\[\QQ = \{\frac{p}{q}:p,q \in \ZZ, q \neq 0\} \]

Cabe recalcar que a la operación inversa a la multiplicación se le conoce 
como \textbf{división}. Cada vez que dividimos entre un número, 
realmente estamos multiplicando por el inverso aditivo del número. 
El operador de la división es $\frac{(...)}{(...)}$, $\div$, /.

La división es una operación con las que se debe tener cuidado, 
pues 

\begin{moral}
    NO PUEDES DIVIDIR ENTRE $0$.
\end{moral}

Antes de continuar, revisemos dos propiedades elementales 
de la suma y la multiplicación:

\textbf{La suma y producto son operaciones conmutativas.} 
$a+b=b+a$ y $ab=ba$.

Esto es la famosísima:

\begin{moral}
    El orden de los factores no altera el producto
\end{moral}

\textbf{La suma y producto son operaciones asociativas.} 
$(a+b)+c=a+(b+c)$ y $(ab)c=a(bc)$.

Sigamos.

Con forme empezaron a usarse aun más las matemáticas formalmente, 
se empezaron a presentar varias operaciones de la siguiente forma: 
"Un número $a$ es multiplicado $x$ veces." Por conveniencia, 
decidieron darle nombre a esta operación; le llamaron 
\textbf{elevar a la $x$-ésima potencia} y se denota $a^x$.

\[a^x=\underbrace{a\times a\times \dots \times a \times a}_x\]

Es muy importante recalcar que elevar a la $x$-ésima potencia  
no es una operación conmutativa ni asociativa. 
Lo que si tiene (similarmente a la suma y multiplicación) es 
una operación inversa que es \textbf{sacar la raiz $n$-ésima} 
y se denota $\sqrt[x]{a}$. 
Esto es sacar un número que al multiplicarse por si mismo $n$ 
veces da como resultado lo que está dentro de la raíz. 

Hay varios números que no pertenecen al conjunto de las 
racionales, varias raíces no lo hacen, por dar un ejemplo. 
Sin embargo, todos los números racionales se pueden 
representar como puntos en la recta numérica. Al 
conjunto de los números que se pueden expresar como puntos en 
la recta numérica pero no son racionales se le conoce como el 
conjunto de los números irracionales y se denota por $\QQ^c$.

Más allá, a la unión de estos dos conjuntos se le conoce como 
el conjunto de los números reales y se denota Por

\[\RR = \QQ \cup \QQ^c\]

De momento puedes ver a la recta numérica como el eje $x$ del 
plano cartesiano.

Dados los conjuntos definidos tenemos las siguientes contenciones 

\[\NN \subset \ZZ \subset \QQ \subset \RR\]

\begin{question}
¿Existe algún número que no pertenezca a ninguno de los 
conjuntos que se han mencionado?
\end{question}

\section{Constitución de los Estados Unidos Algebráicos}

Toda nación soberana se rige por normas. Veamos las principales 
leyes para asegurar que pases una agradable estadía en el 
país del Álgebra.

\subsection{Leyes de los Signos}

\textbf{... en sumas y restas.} Si son signos iguales se suman 
y se pone el signo que tienen ambos. Si son signos diferentes, 
se restan y se pone el signo de la cantidad mayor.

\begin{example}
    $28+11 =$ ?;
    $-9-12 =$ ?;
    $4-1 =$ ?;
    $-57+5 =$ ?.
\end{example}

$39$; $-21$; $3$; $-52$. 

\textbf{... en multiplicaciones y divisiones.} Si son signos 
iguales, el signo del resultado será positivo. Si son signos 
diferentes, el signo del resultado será negativo.

\begin{example}
    $28\times 11 =$ ?;
    $-9\times -12 =$ ?;
    $4\times -1 =$ ?;
    $-57\times 5 =$ ?.
\end{example}

$308$; $108$; $-4$; $-285$. 

Una forma fácil de entenderla las leyes de los signos en 
multiplicaciones y divisiones es la siguiente:

\begin{figure}
    \centering
    \includegraphics[height=8cm]{signlaw.jpg}
    \caption{Cuida con quien te relacionas.}
\end{figure}

\subsection{Jerarquía de Operaciones}

\begin{enumerate}
    \ii Al evaluar una expresión, primero hay que hacer las 
    operaciones dentro de los signos de agrupación (paréntesis (), 
    corchetes [] y llaves${}$), si están unos dentro de otros, 
    se evaluan los paréntesis "más encerrados" 
    hasta llegar a los paréntesis "de afuerita";

    \ii después, son evaluadas las potencias y raíces;

    \ii continúa con las multiplicaciones y las divisiones;

    \ii al final, se hacen las sumas y las restas.
\end{enumerate}

\begin{example}
    Evalua: 
    \[2\cdot\left(2+3\cdot\left[4+5\cdot\left(1+2\right)^2+1\right]\right) \]
\end{example}

Se resuelve primero la suma entre los paréntesis de dentro ($3$); 
el resultado se eleva al cuadrado ($9$); 
se multiplica por 5 ($45$); 
se continua con las operaciones del corchete sumándole 4 y 1 ($50$);
se multiplica por tres ($150$);
se le suma 2 que es la última operación para quitar los paréntesis ($152$);
finalmente, se multiplica por 2 dándonos como resultado $304$.

\begin{figure}
    \centering
    \includegraphics[height=8cm]{opjerarchy.jpg}
    \caption{IMPORTANTÍSIMO.}
\end{figure}

\begin{exercise}
    Simplifica jerárquicamente.
    \begin{enumerate}
        \item $-3 + 4 \cdot (-2) - (-5)$
        \item $\frac{-18 + 6 \cdot 5}{-3}$
        \item $(-4)^2 - (3 \cdot -2) + 8 \div (-2)$
        \item $\sqrt{16 - (-8 \div 4) \cdot (-2)}$
        \item $4 \cdot \left(-3 + 2 \cdot (-2)\right) - (-7)$
        \item $\frac{-3^2 + (-6)^2}{-2 \cdot (-1)^3}$
    \end{enumerate}
\end{exercise}

\subsection{Distribución y Factorización}

\begin{example}
    Fer tiene 5 gatos y 7 perros. Cada gato y cada perro tienen 
    4 patas.

    \begin{enumerate}
        \ii ¿Cuántas patas en total tienen los gatos de Fer?
        \ii ¿Cuántas patas en total tienen los perros de Fer?
        \ii ¿Cuántas patas en total tienen los animales de Fer?
    \end{enumerate}
\end{example}

\begin{soln}
    Cuenta primero los animales, son $5+7 = 12$. Cada animal 
    tiene $4$ patas $\implies$ son $12\times 4$ patas en total.
\end{soln}

\begin{soln}
    Cuenta las patas por separado, son $5\times 4 = 20$ y $7\times 4 = 28$. 
    Sumamos ambos y nos dan $48$ patas en total.
\end{soln}

Si ponemos ambos métodos juntos vemos que 
\[(5+7) \times 4 = 5\times 4 + 7 \times 4\].

Eso es un ejemplo de la

\textbf{Ley distributiva.}

\[a(b+c) = ab+ac\]

Una forma sencilla de verla es yendonos al origen de la 
multiplicación que ya revisamos.

Un caso especial es cuando $a=-1$. Ahí, podemos simplemente 
escribir $-(b+c)$ Cuando expandemos con la Ley distributiva 
en este caso, ¡hay que poner 
atención a los signos!

La ley distributiva también es muy útil cuando la usamos al 
revés.

\begin{example}
    Hay $9$ hijos en la familia $Rosales$. Los padres 
    compran $99$ peces dorados y $72$ peces ángel. Si cada 
    hijo recibe el mismo número de peces, ¿cuántos peces recibirá 
    cada uno?
\end{example}

\begin{soln}
    Podríamos contar todos los peces en total y después 
    dividirlo entre 9; pero, $99$ y $72$ son múltiplos de nueve. 
    Por lo que podríamos dividr la cantidad de peces dorados 
    y la cantidad de peces ángel entre nueve por separado y 
    al final sumar ambos resultados. Por esto, cada niño 
    recibe $11$ peces dorados y $8$ peces ángel, entonces hay un 
    total de
    
    \[99+72 = 9\times (11+8) \]
\end{soln}

Si aplicamos la ley distributiva a la inversa, podemos 
sacar un término en común como en el ejemplo anterior, 
a esto se le llama \textbf{factorizar}.

\begin{definition}[Factorizar]
    Es el proceso de sacar un factor común de cada término 
    en una suma y escribimos el resultado como el factor 
    por una suma más simple 
\end{definition}

\[ab+ac = a(b+c)\]
\subsection{Operaciones con Fracciones}

Verémos métodos sencillos de simplificar fracciones.

\textbf{El factor común}
\[\frac{ak}{bk} = \frac{a}{b}\]

Es como si a un número lo multiplicaras por $k$ y luego lo 
dividieses entre $k$, evidentemente al ser operaciones inversas 
resulta en lo mismo y podemos cancelar las $k$'s sin inconvenientes.

\begin{moral}
    Cancelar solo funciona cuando el numerador y denominador 
    son productos $\frac{a+b}{c+b} \neq \frac{a}{c}$.
\end{moral}

\textbf{Suma y Resta}

\[\frac{a}{bk}+\frac{c}{dk} = \frac{ad+cb}{bdk}\]

Usualmente se escoge $k$ como el mínimo común múltiplo de los 
denomminadores. Aunque si gustas $k$ puede ser $1$ y te queda

\[\frac{a}{b}+\frac{c}{d} = \frac{ad+cb}{bd}\]

\textbf{Multiplicación.}
\[\frac{a}{b} \times \frac{c}{d} = \frac{ac}{bd}\]

La división es equivalente pero la revisaremos a parte por 
el método de visualizarla fácilmente de la tortilla.

\textbf{División} (\textit{La famosa ley de la tortilla}) \textbf{.}
\[\frac{\frac{a}{b}}{\frac{c}{d}} = \frac{a}{b} 
\times \frac{d}{c} = \frac{ad}{bc}\]

Extremos con extremos y medios con medios.

\begin{exercise}
    Simplifica las siguientes operaciones.
    \begin{enumerate}
        \ii $\frac{-3}{\frac{-6}{6}}$
        \ii $\left(\frac{-7.5}{3}\right)\frac{6}{5}$
        \ii $\frac{\frac{4^2}{2^2} + 1}{5^3}$
        \ii $\sqrt{12\cdot2 + 1} + \frac{14\cdot 2 + 8}{6}$
        \ii $2\cdot 3^4 + \frac{2\cdot 3^4}{3}$
        \item $(3^2 \cdot 2 + 4 \cdot 5) \left( -\frac{\pi}{2} \div 4 \right) (-3\pi)$
        \ii $1 + \frac{1}{1 + \frac{1}{1 + \frac{1}{1+\frac{1}{1+1}}}}$
    \end{enumerate}
\end{exercise}

\subsection{Leyes de los Exponentes}

\begin{enumerate}
    \ii $a^1 = a$
    \ii $a^mb^m = (ab)^m$
    \ii $\frac{a^m}{b^m} = \left(\frac{a}{b}\right)^{m}$
    \ii $a^m \times a^n = a^{m+n}$
    \ii $\frac{a^m}{a^n} = a^{m-n}$
    \ii $a^0 = 1$
    \ii $\left(\frac{a}{b}\right)^{-m} = \left(\frac{b}{a}\right)^{m}$
    \ii $a^{\frac{n}{m}} = \sqrt[m]{a^n}$   
\end{enumerate}

Los siguientes dos resultados son muy útiles y se obtienen 
directo de las leyes 7 y 8 de los exponentes, análogamente

\[a^{-m} = \frac{1}{a^m}\]
\[a^{\frac{1}{m}} = \sqrt[m]{a}\]

\begin{exercise}
    Simplifica lo más posible.

    \begin{enumerate}
        \item $(7x y7)^0$
        \item $a^4 \cdot a^6 \cdot a$
        \item $(6x^{10})(3x^4)$
        \item $(3a^2)(-5a^3)$
        \item $(-4a^2 b^3)(-3ab)$
        \item $\frac{x^{16}}{x^{-10}}$
        \item $(7x^{-3}y^{-8})(2x^5y^5)$
        \item $\frac{6x^4y^7}{12x^5y^{-8}}$
        \item $\left(\frac{3x^2y^3}{2xy^4}\right)^3$
        \item $\left(2a^{-2}b^3c^{-1}\right)^{-4}$
    \end{enumerate}
\end{exercise}

Mucho ojo con cumplirlas al pie de la letra eh, luego te meten 
al botellón.

\section{Ecuaciones}

Si, aun más teoría; no te preocupes, esta sección 
es cortita.

\begin{definition}[Ecuación]
    Igualdad entre dos expresiones algebráicas que contiene 
    al menos una incógnita
\end{definition}

Si a las dos partes de una ecuación les efectuamos la misma 
operación, la igualdad se mantiene. Podemos ver a las 
ecuaciones como una balanza.

Si $c$ es cualquier número y $a=b$, entonces todas las 
siguientes ecuaciones son ciertas:

\begin{enumerate}
    \ii $a+c=b+c$
    \ii $a-c=b-c$
    \ii $ac = bc$
    \ii $\frac{a}{c} = \frac{b}{c}$
    \ii $a^c = b^c$
    \ii $\sqrt[c]{a}= \sqrt[c]{b}$
\end{enumerate}

En el punto $4$, $c$ debe ser distinta de $0$ evidentemente. 

Si $a=b$, $c=d$, entonces todas las 
siguientes ecuaciones son ciertas:

\begin{enumerate}
    \ii $a+c=b+d$
    \ii $a-c=b-d$
    \ii $ac = bd$
    \ii $\frac{a}{c} = \frac{b}{d}$
\end{enumerate}

En el punto $4$, $c$ y $d$ deben ser distintos de $0$ evidentemente.

\textbf{Propiedad transitiva.}

Si $a=b$, $b = c \implies a = c$. 

\begin{moral}
    No hagas las operaciones más difíciles de lo que necesitan 
    ser.
\end{moral}

Utiliza todo lo que hemos visto para facilitar los cálculos.

\begin{moral}
    Para encontrar las $n$ variables de un sistema de 
    ecuaciones, ocupas $n$ ecuaciones no inconsistentes. 
\end{moral}

\begin{definition}[Grado de una ecuación]
    Es dado por el mayor exponente entre las incógnitas.
\end{definition}

\subsection{Ecuaciones Lineales de una Variable}

Una ecuación con grado $1$ es llamada \textit{ecuación lineal}

Para resolver ecuaciones lineales de una variable, 
despejamos la incógnita con las estrategias de manipulación 
que hemos visto a lo largo del capítulo.

\begin{example}
    Mi hermana tiene $3$ libros menos que yo. Si ella 
    tiene $6$ libros, ¿cuántos tengo yo?
\end{example}

\begin{soln}
    Planteamos la ecuación $x-3 = 6 \iff x = 9$.
\end{soln}

\begin{exercise}
    Resuelve las siguientes ecuaciones.
    \begin{enumerate}
        \item \( 3t + 9 = -13 \)
        \item \( 5 - \frac{y}{7} = 19 \)
        \item \( 8 - 3x = -6 + 2x \)
        \item \( 2 - y + 10 + \frac{y}{2} = 3y - \frac{7}{3} + 2y \)
    \end{enumerate}
\end{exercise}

\begin{moral}
    La clave para resolver problemas del mundo real 
    es convertir las palabras a matemáticas.
\end{moral}

\newpage

\section{Problemas}

Cada problema que resuelvas te dará el número de treboles 
que especifica ($x \clubsuit$), ¡colecta los más que puedas!

\epigraph{Todos los animales son iguales, pero algunos son 
más iguales que otros.}
{George Orwell, en \emph{Rebelión en la Granja}}

\begin{problem}[$1 \clubsuit$]
    Calcula el valor de \( n \) para que la igualdad 
    \( 3^{100} 3^n = \frac{1}{9} \) sea válida.
\end{problem}

\begin{problem}[$1 \clubsuit$]
    Calcula el valor de \( n \) para que la igualdad 
    \( \frac{2^{100}}{2^n} = 8 \) sea válida.
\end{problem}

\begin{problem}[$1 \clubsuit$]
    Calcula el valor de \( m \) para que la igualdad 
    \( 4^{50} = 2^m \) sea válida.
\end{problem}

\begin{problem}[$1 \clubsuit$]
    ¿A qué potencia hay que elevar \( 4^4 \) para 
    obtener \( 8^8 \)?    
\end{problem}

\begin{problem}[$2 \clubsuit$]
    En cierto planeta hay tantos días en una semana como 
    semanas en un mes como meses en un año. Si un año tiene 
    1331 días, ¿cuántos días tiene cada semana?
\end{problem}

\begin{problem}[$2 \clubsuit$]
    Si \( 16^{2x} = 2^{2x+5} - 2^{2x+4} \), calcula el 
    valor de \( x \).
\end{problem}

\begin{problem}[$2 \clubsuit$]
    El número \( 25^{64} \cdot 64^{25} \) es el cuadrado de un 
    entero positivo \( n \). ¿Cuál es la suma de los dígitos de 
    \( n \)?
\end{problem}

\begin{problem}[$2 \clubsuit$]
    En una reunión hay doble número de mujeres que de hombres y 
    triple número de niños que de hombres y mujeres juntos. 
    ¿Cuántos hombres, mujeres y niños hay si la reunión la 
    componen 96 personas?
\end{problem}

\begin{problem}[$2 \clubsuit$]
    $5$ números consecutivos suman $140$. 
    ¿Cuáles son esos números?
\end{problem}

\begin{problem}[$2 \clubsuit$]
    Ahora soy $3$ años más joven que el doble de mi edad de 
    hace $6$ años. ¿Cuántos años tengo?
\end{problem}

\begin{problem}[AMC 2010, $2 \clubsuit$]
    A Rodrigo le da algebritis y debe tomar una píldora verde 
    y otra rosa diario durante dos semanas. Una pildora 
    verde cuesta $1$ más que la pildora rosa, y las pastillas de 
    Rodrigo costaron en total $546$ por las dos semanas. 
    ¿Cuántos cuesta una píldora verde? 
\end{problem}

\begin{problem}[AMC 2010, $2\clubsuit$]
    El maestro le pidió a Marcela que le restara tres a un 
    número en particular y luego lo dividiera entre 9. En lugar de eso, 
    Marcela la restó 9 y luego lo dividió entre 3 dándole como 
    resultado 43. ¿Cuál era la respuesta correcta?
\end{problem}

\begin{problem}[AHSME, $2\clubsuit$]
    Al final de 1994, Walter tenía la mitad de años que tenía su 
    abuela. La suma de los años en que nacieron ambos es 3838. 
    ¿Cuántos años tendrá Walter al final de 1999?
\end{problem}

\begin{dproblem}[$5 \clubsuit$]
    ¿Cuánto es $1+2+\dots+2024+2025$?
\end{dproblem}

\begin{dproblem}[$4 \clubsuit$]
    Encuentra el valor de $x$, $3\sqrt{x}-2=30-\sqrt{x}$.
\end{dproblem}

\begin{problem}[$4 \clubsuit$]
    Encuentra el valor de $z$, $\cbrt{2z+1}-5+2\cbrt{2z+1}
    =-14$.
\end{problem}

\begin{problem}[$3 \clubsuit$]
    Encuentra el valor de $x$, $\frac{x}{x-1}+\frac{2}{3} = 
    \frac{2}{x-1}$.
\end{problem}

\begin{problem}[AMC 2012, $2\clubsuit$]
    Los miembros de la Liga Rockham de futbol compran calcetines 
    y playeras. Cada par de calcetines cuesta $4$ y las playeras 
    cuestan $5$ más que un par de calcetines. Cada miembro 
    necesita 2 pares de calcetines y dos playeras. Si en total, 
    gastaron $2366$, ¿cuántos miembros hay en la liga?
\end{problem}

\begin{problem}[$4 \clubsuit$]
    Encuentra $y$, si $\sqrt[4]{y}+\sqrt[4]{16y}-2=4$
\end{problem}

\begin{problem}[UNCC, $6 \clubsuit$]
    ¿Qué valores de $x$ satisfacen 
    $\frac{\sqrt{x+1}+\sqrt{x-1}}{\sqrt{x+1}-\sqrt{x-1}}=3$?
\end{problem}

\begin{sproblem}[$7 \clubsuit$]
    \jp

    Para $a \in \NN$, encuentra el valor de 
    $\sqrt{a\sqrt{a\sqrt{a\dots}}}$
\end{sproblem}

\begin{sproblem}[$7 \clubsuit$]
    \jp

    Para $a,b \in \NN$, encuentra el valor de 
    $\sqrt{a\sqrt{b\sqrt{a\sqrt{b\dots}}}}$
\end{sproblem}

\noindent El máximo número de $\clubsuit$ en este capítulo es de $64 \clubsuit$.

\subsection{Los Rifados}



Hubo un empate entre dos equipos, ambos con $21 \clubsuit$.

\vspace{3mm}

\begin{center}
    \begin{tabular}{|c|c|}
        \hline
        Diego & Alondra \\
        Johan & Javier \\
        Mateo & Ximena \\
        Santiago & Yameli \\
        \hline
    \end{tabular}
\end{center}

\vspace{3mm}

\begin{moral}
    \textbf{Gael} fue el \textit{máximo tronador de problemas}, 
    sacando un total de $35 \clubsuit$.
\end{moral}
